\newcommand{\summedSK}{s_{\sum}}

\section{Proposed Solution}
We will specifically focus on improving privacy and integrity in decentralized machine learning where data is additively used as in Ref.~\cite{stevens2021efficientdifferentiallyprivatesecure}.
Instead of using LWE samples as one-time pads though, we will use the output of HPRFs which can in turn be constructed from LWE/ LWR~\footnote{Specifically we require \emph{weak} HPRFs so that the keys are leakage resilient}.
The key advantage is that we only need to do a verifiable setup process \emph{once} for each party and then the party can release data essentially an unlimited number of times (up to a polynomial/ sub-exponential number of times depending on assumptions).
We sketch the protocol in \cref{fig:protSimp}.

\begin{figure}[H]
	\begin{mdframed}
		Setup: \begin{itemize}
			\item Each party $i \in [n]$ generates a secret key $s_i.$
			\item Using MPC, the parties compute $\summedSK = \sum s_i.$
		\end{itemize}
		Aggregation for time step $t$: \begin{itemize}
			\item Each party $i$ calls $\vec{r}_i = F(s_i, t)$  where $F$ is the HPRF.
			\item Each party generates noise $\vec{x}_i$ with standard deviation $O(1/\sqrt{n})$ and publishes data $\overrightarrow{ct}_i = \vec{v}_i + \vec{r}_i + \vec{x}_i$ 
			\item To decode the data, a third party does $\sum_i \overrightarrow{ct}_i - F(\summedSK, t) = \sum_i \vec{v}_i + \sum_i \vec{x}_i$
		\end{itemize}
	\end{mdframed}
	\caption{Outline of the protocol without any verifiability}
	\label{fig:protSimp}
\end{figure}
% \sam{Might need to make the standard deviation larger $O(1/\sqrt{n})$ to make it work as per Prop. 3.1. Should it also
% depend on the number of honest parties???}

% \subsubsection*{Correctness}
% \lev{TODO: mostly straightforward, but need to be careful due to wrap around}

% \subsubsection*{Soundness and Privacy}

% Before proving privacy, we have to recall a theorem about discrete Gaussian noise.


% \begin{proposition}[Sum of Discrete Gaussians, \cite{kairouz2021distributed}]
% 	\label{prop:disc}
% 	Let $\sigma \geq \frac{1}{2}$.
% 	Let $X_{i,j} \sim \mathcal{N}_\mathbb{Z}(0, \sigma^2)$ independently for each $i$ and $j$.
% 	Let $X_i = (X_{i,1}, \dots, X_{i,d}) \in \mathbb{Z}^d$.
% 	Let $Z_m = \sum_{i=1}^m X_i \in \mathbb{Z}^d$. Then, for all $\Delta \in \mathbb{Z}^d$ and all $a \in [1, \infty)$,
% 	\[
% 		D_a(Z_m \parallel Z_m + \Delta) \leq \frac{a \|\Delta\|_2^2}{2m\sigma^2} + \tau_{m,\sigma} d,
% 	\]
% 	where $\tau_{m,\sigma} := 10 \cdot \sum_{k=1}^{m-1} e^{-2\pi^2\sigma^2k/(k+1)}$.
% \end{proposition}


% \begin{theorem}[Preservation of Differential Privacy]
% 	Let $v^\ast \coloneqq \max_{i \in[n]} \|\vec{v}_i\|_2^2$.
% 	Suppose there are at least $m = \theta n$ honest parties where $\theta = \frac{a v^\ast}{2\sigma^2 \epsilon}$ for some $\epsilon > 0$ and $a > 1$.
% 	Assuming that the HPRF is secure and the usage of the fixed-point representation trick in Ref.~\cite{stevens2021efficientdifferentiallyprivatesecure}, the protocol outlined in \cref{fig:protSimp} is $(a, \epsilon)$-RDP.
% 	% \lev{TODO: exact parameters!}
% 	\sam{The exact parameters provided assumes that $\tau d$ from Prop. 3.1 is vanishingly small. In Ref.~\cite{stevens2021efficientdifferentiallyprivatesecure}, they make the same assumption because of the fixed-point representation of the noisy gradients which they used.}
% \end{theorem}
% \begin{proof}[Proof Sketch]
% 	By \cref{prop:disc}, we have that as long as long as $\theta n$ parties are not malicious and add discrete Gaussian noise honestly, the sum of the noisey shares satisfies Renyi differential privacy.
% 	Then, by the data processing inequality, any function of the sum of the honest shares also satisfies Renyi differential privacy.
% \end{proof}

\begin{theorem}[Liveness]
	The protocol offers no liveness guarantees: any $1$ party can prevent the protocol from completing.
\end{theorem}
\begin{proof}[Proof Sketch]
	To see why this is the case, any party, $i$, can simply abort and not publish anything.
	Then, without $s_i$, $A \cdot s_j + e_j$ for $j \neq i$ remains indistinguishable from random.
\end{proof}


\section{Adding in Verifiability}
Because each party's evaluation of the HPRF is relatively inexpensive and simple (via a matrix multiply and rounding), we can use zk-SNARKs to verify that the party's published data is correctly masked relative to the HPRF, that the underlying data is within a certain range (too ensure that the data is not poisoned), and that the noise is not too large (also preventing poisoning).
As a nice benefit, adding verifiability gives a sort of ``traitor-tracing'' mechanism to the system for $N - 1$ parties while assuming that we have $N /2$ honest parties.
\footnote{Traitor tracing is specifically useful when using ``carrot and stick'' incentives to ensure that parties are honest (such as proof-of-stake).}

%Though not within scope, given the uniformity of each proof, a proof aggregator could also be used to enhance practicality.
We sketch the protocol in \cref{fig:prot}.


\begin{figure}[H]
	\begin{mdframed}
		Setup: \begin{itemize}
			\item Each party $i$ generates a secret key $s_i$
			\item Using MPC, the parties compute $\sum s_i = \summedSK$
			\item Each party releases $Comm(s_i)$ where $Comm$ is a randomized commitment function
		\end{itemize}
		Aggregation for time step $t$: \begin{itemize}
			\item Each party $i$ calls $b_i = F(s_i, t)$  where $F$ is the HPRF
			\item Each party generates (quantized) Gaussian noise $\eta_i \in \Z^n$ with standard deviation $O(1/\sqrt{n})$.
			\item  For data $v_i$ and $PRF$ output $b_i$,
				each party publishes data $ct_i = \lfloor Q \cdot (v_i + b_i) \rfloor  + \eta_i$ for quantization/ offset parameter $Q$.
			\item Each party publishes a proof that (1) $b_i$ is generated with key $s_i$ relative to $Comm(s_i)$, (2) $b_i = F(s_i, t)$, and (3) $v_i + \eta_i$ are within some bounds specified by the protocol
			\item To decode the data, a third party checks all the proofs and does $\left(\sum_i ct_i - F(s_{\sum}, t)\right) / Q = \sum_i v_i + \sum \eta_i$
		\end{itemize}
	\end{mdframed}
	\caption{Outline of the final protocol}
	\label{fig:prot}
\end{figure}

\section{Specific Implementation Details}
In this section, we provide some specific implementation details for the protocol outlined in \cref{fig:prot}.

\subsubsection*{HPRF}
To implement the HPRF, we use a very simply LWE/ LWR-based construction similar to that in Ref.~\cite{boneh2013key}.
Unlike in Ref.~\cite{boneh2013key}, we will use the \emph{random oracle model} to dramatically simplify the construction.

Specifically, we will use the following construction for $F(s, t)$:
\begin{enumerate}
	\item Let $n, m, q$ be learning with errors/ learning with rounding parameters.
		Let $p = \lfloor q / 4 \rfloor$
	\item Use random oracle $H$ to generate a matrix $A \in \Z_q^{n \times m}$
	\item Return $\lfloor A \cdot s \rfloor_{\mod p}$ where $\lfloor \cdot \rfloor_{\mod p}$ is the rounding operation modulo $p$.
\end{enumerate}

\subsubsection*{MPC}
To compute the aggregated secret key, we can use a simple MPC protocol.
Specifically, we use Jiff, an MPC Javascript native library, to compute the sum of the secret keys.
To improve efficiency, we use \emph{binary secrets} for the key of the homomorphic PRF.
As pointed out by Ref.~\cite{micciancio2018hardness}, binary-key learning with errors (resp. binary-key learning with rounding) is secure against PPT adversaries if the underlying LWE (resp. LWR) assumption holds.
Otherwise, we use the simple ``sadd'' functionality in Jiff to compute the sum of the secret keys.
Unfortunately, Jiff requires a coordinating server to run the MPC protocol which adds a single point of failure to the setup protocol.
We note though that setup only needs to be run once and the server can be run by a trusted third party.

\subsubsection*{Proofs}
We use Circom and SnarkJS with Groth16 \cite{groth2016size} to generate the zk-SNARKs.

Specifically, we have the following optimizations:
\begin{itemize}
	\item Rather than use a \emph{deterministic} HPRF, we use learning with errors instead of learning with rounding.
		Specifically, we add random ternary (in $\set{-1, 0, 1}$) noise to $A \cdot s$.
		To recover a determistic output, we have the aggregator round the summed output to the nearest multiple of $p$.
		This trick allows us to avoid the need to check rounding in the zk-SNARK and rather just check that the noise is in $\set{-1, 0, 1}$ which requires far fewer constraints.
	\item For our commitments, we use the MiMC hash function as the hash function is more friendly to arithmetic circuits than more standard hash functions (like SHA-256) and thus requires fewer constraints.
\end{itemize}
Setting $n, m = 80$ and $q$ equal to the Circom modulus,
\footnote{
	The Circom modulus is $21888242871839275222246405745257275088548364400416034343698204186575808495617$.
	Technically, having such a big modulus with ternary noise may cause security concerns and seems quite inefficient.
	Future work could ``pack'' multiple multiplications over the LWE modulus into a single Circom multiplication to save on constraints.
	Doing so would also make the storage overhead of the protocol smaller.
}
the circuit has 104,000 constraints and takes approximately \emph{11 seconds} to generate a proof on a T470s thinpad with an i7-7600U processor.

\subsubsection*{Remainder of the Protocol}
We use Node.js and Javascript to implement the remainder of the protocol.
Specifically, we have seperate functionality to (1) run a server to coordinate MPC setup (2) setup each party (3) generatea proof and data share for each party at time step $t$ (4) aggregate the data shares and verify the proofs.
Throughout the implementation, we also use a simple JSON-based database to store the secret keys, commitments, and data.

\section{Future Work}
Besides practical optimizations to the protocol, there are several directions for future work.
Specifically, our work does not have any interesting liveness guarantees.
If one party aborts, the protocol cannot complete!
So, we would hope to use \emph{correlated pseudo-random functions} as seen in Ref.~\cite{boyle2020correlated} to create a sort of $k$-wise secrecey mask: as long as $k$ out of $n$ parties reveal there shares, the protocol can complete and recover the sum of the data for those $k$ parties.


\newcommand{\summedSK}{s_{\sum}}

\section{Proposed Solution}
We will specifically focus on improving privacy and integrity in decentralized machine learning where data is additively used as in Ref.~\cite{stevens2021efficientdifferentiallyprivatesecure}.
Instead of using LWE samples as one-time pads though, we will use the output of HPRFs which can in turn be constructed from LWE/ LWR~\footnote{Specifically we require \emph{weak} HPRFs so that the keys are leakage resilient}.
The key advantage is that we only need to do a verifiable setup process \emph{once} for each party and then the party can release data essentially an unlimited number of times (up to a polynomial/ sub-exponential number of times depending on assumptions).
We sketch the protocol in \cref{fig:protSimp}.

\begin{figure}[H]
	\begin{mdframed}
		Setup: \begin{itemize}
			\item Each party $i \in [n]$ generates a secret key $s_i.$
			\item Using MPC, the parties compute $\summedSK = \sum s_i.$
		\end{itemize}
		Aggregation for time step $t$: \begin{itemize}
			\item Each party $i$ calls $\vec{r}_i = F(s_i, t)$  where $F$ is the HPRF.
			\item Each party generates noise $\vec{x}_i$ with standard deviation $O(1/n)$ and publishes data $\overrightarrow{ct}_i = \vec{v}_i + \vec{r}_i + \vec{x}_i$ 
			\item To decode the data, a third party does $\sum_i \overrightarrow{ct}_i - F(\summedSK, t) = \sum_i \vec{v}_i + \sum_i \vec{x}_i$
		\end{itemize}
	\end{mdframed}
	\caption{Outline of the protocol without any verifiability}
	\label{fig:protSimp}
\end{figure}
\sam{Might need to make the standard deviation larger $O(1/\sqrt{n})$ to make it work as per Prop. 3.1. Should it also
depend on the number of honest parties???}

\subsubsection*{Correctness}
\lev{TODO: mostly straightforward, but need to be careful due to wrap around}

\subsubsection*{Soundness and Privacy}

Before proving privacy, we have to recall a theorem about discrete Gaussian noise.


\begin{proposition}[Sum of Discrete Gaussians, \cite{kairouz2021distributed}]
	\label{prop:disc}
	Let $\sigma \geq \frac{1}{2}$.
	Let $X_{i,j} \sim \mathcal{N}_\mathbb{Z}(0, \sigma^2)$ independently for each $i$ and $j$.
	Let $X_i = (X_{i,1}, \dots, X_{i,d}) \in \mathbb{Z}^d$.
	Let $Z_m = \sum_{i=1}^m X_i \in \mathbb{Z}^d$. Then, for all $\Delta \in \mathbb{Z}^d$ and all $a \in [1, \infty)$,
	\[
		D_a(Z_m \parallel Z_m + \Delta) \leq \frac{a \|\Delta\|_2^2}{2m\sigma^2} + \tau_{m,\sigma} d,
	\]
	where $\tau_{m,\sigma} := 10 \cdot \sum_{k=1}^{m-1} e^{-2\pi^2\sigma^2k/(k+1)}$.
\end{proposition}


\begin{theorem}[Preservation of Differential Privacy]
	Let $v^\ast \coloneqq \max_{i \in[n]} \|\vec{v}_i\|_2^2$.
	Suppose there are at least $m = \theta n$ honest parties where $\theta = \frac{a v^\ast}{2\sigma^2 \epsilon}$ for some $\epsilon > 0$ and $a > 1$.
	Assuming that the HPRF is secure and the usage of the fixed-point representation trick in Ref.~\cite{stevens2021efficientdifferentiallyprivatesecure}, the protocol outlined in \cref{fig:protSimp} is $(a, \epsilon)$-RDP.
	% \lev{TODO: exact parameters!}
	\sam{The exact parameters provided assumes that $\tau d$ from Prop. 3.1 is vanishingly small. In Ref.~\cite{stevens2021efficientdifferentiallyprivatesecure}, they make the same assumption because of the fixed-point representation of the noisy gradients which they used.}
\end{theorem}
\begin{proof}[Proof Sketch]
	By \cref{prop:disc}, we have that as long as long as $\theta n$ parties are not malicious and add discrete Gaussian noise honestly, the sum of the noisey shares satisfies Renyi differential privacy.
	Then, by the data processing inequality, any function of the sum of the honest shares also satisfies Renyi differential privacy.
\end{proof}

\begin{theorem}[Liveness]
	The protocol offers no liveness guarantees: any $1$ party can prevent the protocol from completing.
\end{theorem}
\begin{proof}[Proof Sketch]
	To see why this is the case, any party, $i$, can simply abort and not publish anything.
	Then, without $s_i$, $A \cdot s_j + e_j$ for $j \neq i$ remains indistinguishable from random.
\end{proof}


\section{Adding in Verifiability}
Because each party's evaluation of the HPRF is relatively inexpensive and simple (via a matrix multiply and rounding), we can use zk-SNARKs to verify that the party's published data is correctly masked relative to the HPRF, that the underlying data is within a certain range (too ensure that the data is not poisoned), and that the noise is not too large (also preventing poisoning).
As a nice benefit, adding verifiability gives a sort of ``traitor-tracing'' mechanism to the system for $N - 1$ parties while assuming that we have $N /2$ honest parties.
\footnote{Traitor tracing is specifically useful when using ``carrot and stick'' incentives to ensure that parties are honest (such as proof-of-stake).}

%Though not within scope, given the uniformity of each proof, a proof aggregator could also be used to enhance practicality.
We sketch the protocol in \cref{fig:prot}.


\begin{figure}[H]
	\begin{mdframed}
		Setup: \begin{itemize}
			\item Each party $i$ generates a secret key $s_i$
			\item Using MPC, the parties compute $\sum s_i = \summedSK$
			\item Each party releases $Comm(s_i)$ where $Comm$ is a randomized commitment function
		\end{itemize}
		Aggregation for time step $t$: \begin{itemize}
			\item Each party $i$ calls $b_i = F(s_i, t)$  where $F$ is the HPRF
			\item Each party generates (quantized) Gaussian noise $\eta_i \in \Z^n$ with standard deviation $O(1/n)$ and scaled by quantization factor $Q$.
			\item  For data $v_i$ and $PRF$ output $b_i$,
				each party publishes data $ct_i = Q \cdot v_i + b_i + \eta_i$
			\item Each party publishes a proof that (1) $b_i$ is generated with key $s_i$ relative to $Comm(s_i)$, (2) $b_i = F(s_i, t)$, and (3) $v_i + \eta_i$ are within some bounds specified by the protocol
			\item To decode the data, a third party checks all the proofs and does $\left(\sum_i ct_i - F(s_{\sum}, t)\right) = \left(\sum_i v_i + \sum \eta_i\right) / Q$
		\end{itemize}
	\end{mdframed}
	\caption{Outline of the final protocol}
	\label{fig:prot}
\end{figure}

% Delta = sum of v_is
% Everything has to be a vector

% Define theta to be some fraction that should be related to epsilon and noise parameter
% CHange n in Prop 3.1 to honest parties

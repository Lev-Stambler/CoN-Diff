\documentclass[11pt]{article}
\usepackage{CSTheoryToolkitCMUStyle}
\usepackage{Custom}
\usepackage{cleveref}
% \usepackage{biblatex}
% \addbibresource{CSTheoryToolkitCMUStyle.bib}
\usepackage{mdframed}



%%%%% Stuff you can change %%%%%%%%%%%%%%%%%%%%%%%%%%%%%%%%%%
\newcommand{\myname}{Lev Stambler}

%%%%% Section-renaming code by egreg
\makeatletter
% we use \prefix@<level> only if it is defined
\renewcommand{\@seccntformat}[1]{%
  \ifcsname prefix@#1\endcsname
    \csname prefix@#1\endcsname
  \else
    \csname the#1\endcsname\quad
  \fi}
% Now we define our homework section prefixes
\makeatother
%%%%%

\begin{document}

\title{CoN Diff: Verified Pseudo-Correlated Noise for Differential Privacy}

\author{\myname\ and Shi Jie Samuel Tan}

\date{}
\maketitle

\newcommand{\lev}[1]{\textcolor{red}{\textbf{Lev: #1}}}
\newcommand{\sam}[1]{\textcolor{green}{\textbf{Sam: #1}}}

%\begin{abstract}
%\end{abstract}

\section{Introduction}
    ``Software is eating the world, but AI is going to eat software'' is a well-known quote by Jensen Huang, CEO of NVIDIA.
    Still, AI is notoriously data-hungry and privacy is a big concern.
    When considering private data---such as financial, medical, and political data---privacy is paramount.
    Unfortunately, privacy is often at odds with data utility and the AI model's robustness against adversarial inputs.
    In this work, we propose a novel approach to privacy-preserving machine learning that primarily combines Differential Privacy (DP) and Learning with Errors (LWE) as seen in Ref.~\cite{stevens2021efficientdifferentiallyprivatesecure}.
    In addition, we add zero-knowledge proofs (ZKPs) to ensure that inputted data is not poisoned with traitor-tracing capabilities.
    ~\footnote{Poisoned data is data that is intentionally manipulated to degrade the performance of the model.
    For example, a model trying to calculate the average height of a group of people could be poisoned by inputting a height of 1000 feet.}
    Moreover, we use new cryptographic tools to allow for \emph{efficient and verifiable} repeated data releases without a repeated setup process.



\section{Preliminaries}

\subsection{Zero-Knowledge Proofs}
\emph{Zero-knowledge proofs} (ZKPs) are cryptographic protocols that allow one party, called the prover, to demonstrate the validity of a statement to another party, the verifier, without revealing any additional information beyond the fact that the statement is true~\cite{goldwasser2019knowledge}. 
In the original formulation by Goldwasser, \emph{et al.}, they explored interactive protocols that uses notions of computational and statistical indistinguishabilities to achieve zero-knowledge verification.
Subsequently, the concept of non-interactive zero-knowledge proofs (NIZKs) was introduced, which allows the prover to generate a proof that can be verified by the verifier without any interaction~\cite{fiat1986prove,blum2019non}.
These works became fundamental for digital signature schemes and cryptographic identification protocols, making zero-knowledge proofs more practical, especially in cases where communication and exchange of information are unrealistic or undesirable.



\subsection{Zero-Knowledge Succinct Non-Interactive Argument of Knowledge}
As non-interactivity becomes more important in modern cryptographic protocols, researchers have developed more efficient and scalable zero-knowledge proofs that do not require interaction between the prover and verifier.
\emph{zk-SNARKs} (Zero-Knowledge Succinct Non-Interactive Argument of Knowledge) are a specific type of ZKP that is particularly efficient and scalable, making them suitable for a wide range of applications, from blockchain to secure messaging~\cite{ben2014succinct}.
The theory behind zk-SNARKs was further formalized by Ben-Sasson, \emph{et al.}, who introduced the concept of a \emph{quadratic arithmetic program} (QAP) to represent computations in a succinct and verifiable manner~\cite{ben2013snarks}.
\emph{Pinocchio} is a popular zk-SNARK construction that uses the QAP framework to generate proofs for NP statements~\cite{parno2016pinocchio}.
Because blockchain systems require efficient and scalable verification of transactions, zk-SNARKs have become a key tool for ensuring privacy and security in decentralized environments~\cite{sasson2014zerocash,kosba2016hawk}.
% \sam{In our work, we use RISC-0 to produce our ZKPs though other ZKPs could be used as well~\footnote{\href{https://risczero.com/}{RISC-0 Homepage}}.}

\subsection{Pseudo-Random Functions}
\emph{Pseudo-random functions} (PRFs) are cryptographic primitives that mimic the behavior of random functions while being efficiently computable.
Some of the well-known PRFs include HMAC, AES, and SHA-256, which are widely used in secure communication protocols and cryptographic applications.
Lattice-based PRFs have also been developed to provide post-quantum security and resistance to quantum attacks~\cite{peikert2014lattice}.
Such PRFs are based on the hardness of lattice problems, such as Learning with Errors (LWE) and Ring-LWE, which are believed to be secure against quantum adversaries~\cite{regev2009lattices, lyubashevsky2013ideal}.
Before we state the definition of LWE, we first provide a metric of quantifying the indistinguishability of two distributions.

\begin{definition}[$(\alpha, \beta)$-indistinguishability]
	Two distributions $D_0$ and $D_1$ are $(\alpha, \beta)$-indistinguishable if for any circuit $C$ of size at most $\alpha$, the following holds:
	\[\left|\Pr[C(D_0) = 1] - \Pr[C(D_1) = 1]\right| \leq \beta.\]	
\end{definition}

\subsubsection{Learning with Errors}
The LWE problem that was intoduced by Regev forms the foundation for the efficient construction of lattice-based PRFs.
It is stated as follows:

\begin{definition}[LWE Problem~\cite{regev2009lattices}]
	A LWE problem is $(\alpha, \gamma, \beta)$-hard if the following two distributions are $(\alpha, \beta)$-indistinguishable:
	\begin{enumerate}
		\item Sample random $\vec{s} \in \Z_q^n$ and output $\gamma$ pairs $\left(\vec{a}_i, b_i\right) \in \Z_q^n \times \Z_p$, where $\vec{a}_i$'s are uniformly random and independent and $b_i = \left\langle \vec{a}_i, \vec{s}\right\rangle + e_i \bmod q$ for small random $e_i$.
		\item Output $\gamma$ uniformly random and independent pairs $\left(\vec{a}_i, b_i\right) \in \Z_q^n \times \Z_p$.
	\end{enumerate}
\end{definition}

\subsubsection{Learning with Rounding}
\emph{Learning with Rounding} (LWR) is a variant of LWE that can be understood as the ``derandomized'' version of the problem.
Instead of adding noise to the inner product, we round $\left\langle \vec{a}_i, \vec{s}\right\rangle$ to the nearest element of a public subset of $p$ well-separated values in $\Z_q$, where $p$ is much smaller than $q$~\cite{bogdanov2017pseudorandom}.
Because there are only $p$ possible rounded values, they can be labeled as elements of $\Z_p$ and denoted as $\left\lfloor\left\langle \vec{a}_i, \vec{s}\right\rangle\right\rceil_p$, where $\lfloor x \rceil_p$ equals $\lfloor (p/q)\cdot x \bmod q\rceil \bmod p$.
The LWR problem can be stated as follows:
\begin{definition}[LWR Problem~\cite{banerjee2012pseudorandom}]
	A LWR problem is $(\alpha, \gamma, \beta)$-hard if the following two distributions are $(\alpha, \beta)$-indistinguishable:
	\begin{enumerate}
		\item Sample random $\vec{s} \in \Z_q^n$ and output $\gamma$ pairs $\left(\vec{a}_i, b_i\right) \in \Z_q^n \times \Z_p$, where $\vec{a}_i$'s are uniformly random and independent and $b_i = \left\lfloor\left\langle \vec{a}_i, \vec{s}\right\rangle\right\rceil_p$.
		\item Output $\gamma$ uniformly random and independent pairs $\left(\vec{a}_i, b_i\right) \in \Z_q^n \times \Z_p$.
	\end{enumerate}
\end{definition}

A useful proposition was introduced by Banerjee, Peikert and Rosen in Ref.~\cite{banerjee2012pseudorandom} that shows that LWR is a weaker assumption than LWE.
Let us define an LWE-error distribution to be \emph{$B$-bounded} if for all errors $e$ in the support of the distribution it holds that $e \in [-B, B]$.
In addition, let $RD$ be the cost of rounding an element in $\Z_q$ to the nearest element in $\Z_p$.

\begin{proposition}[LWE hardness Implies LWR hardness~\cite{banerjee2012pseudorandom}]\label{prop:lwe2lwr}
	If the LWE problem is $(\alpha, \gamma, \beta)$-hard for some $B$-bounded error distribution, then the LWR problem is $(\alpha-\gamma\cdot RD, \gamma, (\gamma p(2B+1)/q)+\beta)$-hard.
\end{proposition}

% \subsubsection{Non-uniform Learning with Errors}
% While the assumption of LWE hardness is already widely used in lattice-based cryptography, the uniformness assumption of the $\vec{a}$ vector ultimately limits the application of LWE for a greater range of practical problems.
% In Ref.~\cite{boneh2013key}, Boneh, Lewi, Montgomery, and Raghunathan introduced the concept of \emph{non-uniform LWE} to address this limitation.
% To make an explicit distinction between the uniform and non-uniform LWE problems, we introduce the distribution $\eta$ from which $\vec{a}_i' \in \Z_q^{n'}$ is sampled from whereas the vector $\vec{a}_i \in \Z_q^n$ is sampled uniformly at random. 

% \begin{definition}[Non-uniform LWE Problem~\cite{boneh2013key}]
% 	A non-uniform LWE problem is $(\alpha, \gamma, \beta)$-hard if the following two distributions are $(\alpha, \beta)$-indistinguishable:
% 	\begin{enumerate}
% 		\item Sample random $\vec{s} \in \Z_q^n$ and output $\gamma$ pairs $\left(\vec{a}_i', b_i\right) \in \Z_q^{n'} \times \Z_p$, where $\vec{a}_i'$'s are randomly and independently chosen from some distribution $\eta$ and $b_i = \left\langle \vec{a}_i', \vec{s}\right\rangle + e_i \bmod q$ for small random $e_i$.
% 		\item Output $\gamma$ uniformly random and independent pairs $\left(\vec{a}_i', b_i\right) \in \Z_q^{n'} \times \Z_p$.
% 	\end{enumerate}
% \end{definition}

% It was shown in Ref.~\cite{boneh2013key} that for some $n' \geq n$, there exists a reduction from LWE to NLWE for specific choices of $\eta$.
% This implies that the NLWE problem is at least as hard as the LWE problem.
% Given suitable parameters, it has been shown that the following distributions are suitable candidates for $\eta$:
% \begin{enumerate}
% 	\item The uniform distribution over $\Z_2^{n'}$ for sufficiently large $n'$,
% 	\item a discrete Gaussian distribution over $\Z^{n'}$ with a sufficiently large $n'$ and standard deviation $\sigma$.
% 	\item a uniform distribution over a sufficiently large linear subspace $V$ of $\Z_q^{n'}$.
% \end{enumerate}



\subsubsection{Homomorphic Pseudo-Random Functions}
LWR is particularly useful for constructing efficient and secure cryptographic primitives, such as \emph{homomorphic pseudo-random functions} (HPRFs)~\cite{boneh2013key}.
In this paper, we focus on HPRFs that are homomorphic over their keys under the addition operation.
Such HPRFs are also referred to as \emph{key-homomorphic PRFs} and were originally introduced by Naor, Pinkas, and Reingold in Ref.~\cite{naor1999distributed} where these PRFs were constructed using random oracles and the Decisional Diffie-Hellman hardness assumption.
\begin{definition}[Key-Homomorphic PRFs~\cite{boneh2013key}]
	Consider an efficiently computable function $F: \calK \times \calX \to \calY$ such that $(\calK, \oplus)$ and $(\calY, \otimes)$ are both groups.
	We say that the tuple $(F, \oplus, \otimes)$ is a \emph{key-homomorphic} PRF (KHPRF) if the following two properties hold:
	\begin{enumerate}
		\item $F$ is a secure pseudorandom function.
		\item For every $k_1, k_2 \in \calK$ and every $x \in \calX$, $F\left(k_1, x\right) \otimes F\left(k_2, x\right) = F\left(k_1 \oplus k_2, x\right).$
	\end{enumerate}
	\end{definition}

Boneh \emph{et al.} provided the first construction of an (almost) key-homomorphic PRF without random oracles by basing their construction on the LWE problem.
% By using ideas from LWE-based PRFs constructed by Banerjee, Peikert, and Rosen in Ref.~\cite{banerjee2012pseudorandom} and the reduction from LWE to LWR shown in Proposition~\ref{prop:lwe2lwr}, Boneh \emph{et al.} constructed an (almost) key-homomorphic PRF that has a parameter $\gamma$ that corresponds to the error allowed in the homomorphism.

% \begin{definition}[$\zeta$-Almost Key-Homomorphic PRFs~\cite{boneh2013key}]\label{def:akhprf}
% 	Let $F:\calK \times \calX \to \Z_p^m$ be an efficiently computable function such that $(\calK, \oplus)$ is a group.
% 	We say that the tuple $(F, \oplus)$ is a $\zeta$-almost key-homomorphic PRF ($\gamma$-AKHPRF) if the following two properties hold:
% 	\begin{enumerate}
% 		\item $F$ is a secure pseudorandom function.
% 		\item For every $k_1, k_2 \in \calK$ and every $x \in \calX$, there exists a vector $\vec{e} \in [0, \zeta]^{m}$ such that 
% 		\[F\left(k_1, x\right) + F\left(k_2, x\right) = F\left(k_1 \oplus k_2, x\right) + \vec{e} \bmod p.\]
% 	\end{enumerate}
% \end{definition}

% Crucially, the construction by Boneh \emph{et al.} uses \emph{seed homomorphic psudorandom generators} (PRGs) to generate the keys for the homomorphic PRFs.

% \begin{definition}[Seed Homomorphic PRGs~\cite{boneh2013key}]
% 	An efficiently computable function $G: \calX \to \calY$, where $(\calX, \oplus)$ and $(\calY, \otimes)$ are groups, is said to be \emph{seed homomorphic} if the following two properties hold:
% 	\begin{enumerate}
% 		\item $G$ is a secure PRG.
% 		\item For ever $s_1, s_2 \in \calX$, we have that $G\left(s_1\right) \otimes G\left(s_2\right) = G\left(s_1 \oplus s_2\right)$.
% 	\end{enumerate}
% \end{definition}

% Similar to the notion of approximate key-homomorphism that has been stated for the PRFs in Definition~\ref{def:akhprf}, we can define a notion of approximate seed-homomorphism for the PRGs by assuming the LWR problem is hard and allowing
% \[G_{LWR}\left(s_1 + s_2\right) = G_{LWR}\left(s_1\right) + G_{LWR}\left(s_2\right) + \vec{e}\]
% where $\vec{e} \in \{0, 1, 2\}^m$ is a small error vector.

% Without going into the details for how we can construct a $\zeta$-AKHPRF from a $\zeta'$-ASHPRG, we provide some intuition for how a KHPRF can be constructed from a seed-homomorphic PRG as stated in Ref.~\cite{boneh2013key}.
% We start off by considering a seed-homomorphic PRG $G: \calK \to \calK \times \calK$ where $(\calK, \oplus)$ is a group.
% Since the output of $G(s)$ is in $\calK \times \calK$, we can write $G_0(s)$ for the left half of $G(s)$ and $G_1(s)$ for the other half. 
% Now, we can construct the KHPRF $F$ with key space $\calK$ and input space $\{0, 1\}^\ell$ as follows:
% \[F(k, x=x_1, \ldots, x_\ell) = G_{x_\ell}\left(G_{x_{\ell - 1}}\left(\ldots G_{x_2}\left(G_{x_1}(k)\right)\ldots \right)\right).\]
% If $G$ is a secure PRG, then $F$ is a secure PRF.
% Moreover, it can be easily checked that if $G$ is seed-homomorphic, then $F$ is a KHPRF.

% Since then, the construction by Boneh \emph{et al.} has been generalized and made more efficient in Refs.~\cite{newKeyHom, kim2020key}.

\subsection{Federated Learning}
\emph{Federated learning} (FL) was first proposed by Google in 2017~\cite{mcmahan2017communication}.
It is a decentralized machine learning paradigm where multiple parties collaboratively train a model without sharing their data.
Each party trains a local model on their data and sends the model updates to a central server, which aggregates the updates to produce a global model while preserving the privacy of the individual data.
The key difference between FL and other distributed optimization methods is that FL does not assume that the training data is IID or has equal-size across the parties. 
FL is particularly useful in scenarios where data privacy is a concern, such as healthcare, finance, and IoT applications~\cite{sheller2020federated, li2020review, yuan2020federated, xu2021federated, nevrataki2023survey, bhatti2024enhancing}.

To be specific, we are interested in the following setting:
\begin{definition}[Federated Learning~\cite{mcmahan2017communication}]
		Let $\calP_1, \ldots, \calP_N$ be $N$ different parties with private datasets $\left\{\calD_i\right\}_{i = 1}^N$.
		Each party $\calP_i$ has a local model $w_i\in \R^{d_i}$ and trains the model on their dataset $\calD_i$.
		The parties collaboratively train a global model $\vec{w} = \left(w_1, w_2, \ldots, w_N\right)$ by aggregating the local models $w_i$ without sharing $\calD_i$.
		The global model $\vec{w}$ is updated iteratively by sending the local model updates to a central server, which aggregates the updates to produce the global model.

		Letting $n$ be the size of the total data possessed by all parties and $\ell_j\left(x_j, y_j; w_i\right)$ be some loss function computed on examples $\left(x_j, y_j\right)$ with model parameters $w_i$, the FL model has a finite-sum objective of the form:
		\[\min_{\vec{w} \in \left(\R^{d_1}, \ldots, \R^{d_N}\right)} f\left(\vec{w}\right)\]
		where $f\left(\vec{w}\right) = \sum_{i = 1}^N \frac{\left|\calD_i\right|}{n} F_i\left(w_i\right)$ and $F_i\left(w_i\right) = \frac{1}{\left|\calD_i\right|} \sum_{\left(x_j, y_j\right) \in \calD_i}\ell_j\left(x_j, y_j;w_i\right).$
		The $F_i$ functions are the local loss functions for party $\calP_i$ and the $\ell_j$ functions are the loss functions for each example.

	\end{definition}

\subsection{Differential Privacy}
\emph{Differential Privacy} (DP) uses noise to mask the contribution of individual data points in statistical analysis, ensuring that sensitive information is protected~\cite{dwork2006differential}. 
By adding random noise from distributions such as Laplace or Gaussian, DP limits the ability of an adversary to infer specific data, even when multiple queries are made, at the expense of the utility of results. 
In certain scenarios, correlated noise, where noise is not independent across data contributors, can improve privacy and utility, though verifying its correct implementation becomes more challenging.
% For a more in-depth look into DP, we recommend a survey on Local Differential Privacy (LDP)~\cite{yang2023local}.
In our work, we use the $(\eps, \delta)$-differential privacy definition, though our ideas can be extended to other definitions.
% We have two flavors of DP where noise is either applied locally or non-locally.
%Intuitively, we can think of differential privacy as being a sort of ``2-wise'' independence condition on the input data: if we replace \emph{one} of the user's piece of data with some other possible piece of data, then differential privacy gaurentees that an adversary \emph{cannot} distinguish between the two datasets up to some probabality.
%Though not cryptographic in nature, differential privacy offers a notion of ``plausible-deniability'' which can also be generalized to a more $k$-wise independent setting \lev{TODO: cite?}.

\begin{definition}[$(\eps, \delta)$-Differential Privacy,~\cite{Bassily_2015}]
	Let $\eps > 0$ and $\calX$ be the data element domain.
	Then, a randomized algorithm, $\calM : \calX^n \rightarrow \calM\left(\calX^n\right)$ that is applied to an $n$-element dataset satisfies $(\eps, \delta)$-differential privacy if and only if for all $\vec{x},\vec{x}' \in \calX^n$ where $\vec{x}'_i \neq \vec{x}_i$ for some $i$ and any $S \subseteq \text{Range}(\calM)$,
	\[
		\Pr[\calM(\vec{x}) \in S] \leq e^\eps \Pr[\calM(\vec{x}') \in S] + \delta.
	\]
\end{definition}

% \lev{Make sure $(\eps, \delta)$-definition is removed elsewhere as well.}
% \sam{do we have to remove it from the earlier definitions? I don't think it's too confusing since the $\epsilon$ and $\delta$ are defined with respect to different things and they are fairly self-contained.}

% \begin{definition}[R\'enyi Divergence, \cite{mironov2017renyi}]
% 	For two probability distributions \( P \) and \( Q \) defined over \( R \), the Rényi divergence of order \( a > 1 \) is defined as:
% 	\[
% 		D_a(P \| Q) =
% 		\frac{1}{a - 1} \log \mathbb{E}_{x \sim Q} 
% 		\left[ 
% 			\left( 
% 				\frac{P(x)}{Q(x)} 
% 			\right)^{a} 
% 		\right],
% 	\]
% 	where \( P(x) \) is the density of \( P \) at \( x \), and all logarithms are natural.

% 	For the endpoints of the interval \( (1, \infty) \), the Rényi divergence is defined by continuity:
% 	\[
% 		D_1(P \| Q) = \lim_{a \to 1} D_a(P \| Q),
% 	\]
% 	which can be verified to be the Kullback-Leibler divergence:
% 	\[
% 		D_1(P \| Q) = \mathbb{E}_{x \sim P} \left[ \log \frac{P(x)}{Q(x)} \right].
% 	\]
% 	Similarly,
% 	\[
% 		D_\infty(P \| Q) = \sup_{x \in \text{supp } Q} \log \frac{P(x)}{Q(x)}.
% 	\]
% \end{definition}

% As pointed out in Ref.~\cite{mironov2017renyi},
% \[
% 	D_\infty(f(P) \| f(P')) \leq \eps
% \]
% for two adjacent inputs $P, P'$ if and only if $f$ satisfies $(\eps, 0)$-differential privacy.
% % \lev{TODO: check on the delta}.
% In fact, we can relax the definition of differential privacy to the following:
% \begin{definition}[$(a, \epsilon)-$R\'enyi Differential Privacy, \cite{mironov2017renyi}]
% 	Let $a > 1$ and $\epsilon > 0$.
% 	A randomized mechanism $f$ satisfies $(a, \epsilon)$-R\'enyi differential privacy ($(a, \epsilon)$-RDP) if for all pairs of adjacent inputs $P$ and $P'$, we have
% 	\[
% 		D_a\left(f(P) \parallel f(P')\right) \leq \epsilon.
% 	\]	
% \end{definition}

% So, similar to Ref.~\cite{stevens2021efficientdifferentiallyprivatesecure}, we will work with the $(a, \epsilon)$-Rényi divergence in our proofs.

% \lev{TODO: remove local diff privacy or keep it?}

%\lev{IDK if this is the actual defn}
% Note that in the above, the randomized algorithm $\calM$ is applied globally.
% We can also deffine differential privacy locally:

% \begin{definition}[$(\eps, \delta)$-Local Differential Privacy,~\cite{Bassily_2015}]
% 	Let $\eps > 0$ and $\calX$ be the domain of the data.
% 	Then, a randomized algorithm, $\calM_{loc} : \calX \rightarrow \calM_{loc}(\calX)$ which is applied to the data independently satisfies $(\eps, \delta)$-local differential privacy if and only if for all pairs $x, x' \in \calX$
% 	for any possible $S \subseteq \text{Range}(\calM_{loc})$, we have
% 	\[
% 		\Pr[\calM_{loc}(x) \in S] \leq e^\eps \Pr[\calM_{loc}(x') \in S] + \delta.
% 	\]
% \end{definition}

DP is usually achieved by adding noise to a dataset.
Though many mechanisms exist, we outline the \emph{Laplace Mechanism} for DP.
Let $f: \mathcal{X}^n \rightarrow \R^d$ be some query associated with the entire $n$-element dataset (e.g. learning the mean of the data).
Then, for any $\vec{x} \in \calX^n$ as outlined in \cite{Bassily_2015} where $N \sim Lap(0, \Delta f/ \eps)^{d}$ where $\Delta f$ is the $\ell_1$ sensitivity of $f$
\[
	\calM(\vec{x}) = f(\vec{x}) + N.
\]
% Analogously, we can think of local-differential privacy as the above but we replace $\vec{x}$ with $x$.

\paragraph{Problem with Differential Privacy:} DP requires trusting some centralized aggregator.
For example, in federated learning, each party would have to send their updated gradients \emph{in the clear} to an aggregator.
Recent work shows that an aggregator can almost perfectly recover the users' data from the gradient delta which yields a large privacy violation~\cite{gupta2022recoveringprivatetextfederated}.
On the other hand, local DP preserves users' privacy but data degradation worsens with the number of users.
Specifically, imagine trying to compute the sum of a dataset with $n$ users.
If each user adds Laplacian noise with a constant standard deviation, $\Delta$, then the sum of the data (calculated by adding all of the noisy inputs from each user) has Laplacian noise with paramater $n \Delta$!


\subsection{Current Techniques}
We can break down privacy-preserving machine learning into two main categories: non-cryptographic (e.g. DP) and cryptographic (e.g. Multi-Party Computation (MPC)).
% \footnote{For the sake of brevity, we will not be discussing Homomorphic Encryption HE) in this outline, but \emph{verifiable} HE it is not quite practical.}


\paragraph{Multi-Party Computation.}
Secure multi-party computation (SMPC)~\cite{yao1986generate} is the other primary privacy preserving mechanism in ML.
We draw from a recent survey on SMPC~\cite{zhou2024secure}.
SMPC is a cryptographic protocol which allows multiple parties to compute a function on their private data without revealing their data to each other.
It can also be integrated with other techniques like cut-and-choose type proofs~\cite{lindell2016fast} to provide additional security guarantees like data integrity checks.
Unfortunately, SMPC is not a panacea.
Specifically, it is very slow with input data integrity checks and cannot scale to large datasets and computations.
Moreover, practical implementations require many multi-party interactions~\cite{zhao2019secure}.

\subsubsection*{Combining MPC, LWE, and Differential Privacy}
Ref.~\cite{stevens2021efficientdifferentiallyprivatesecure} introduced the idea of combining Learning with Error (LWE) with differential privacy for aggregation~\footnote{we discovered this idea independently prior to finding the referenced paper}.
The key idea is to use an LWE sample for each party $i$, $r_i = A \cdot s_i + e_i$ where $s_i$ is secret and $e_i$ is private noise, to mask each users' data.
Then, using MPC or a similar protocol, the users can publish $s_{\sum} = \sum s_i$ which can then be used to compute an approximation of the aggregated one-time pads via $A \cdot s_{\sum} \approx \sum r_i = \sum A \cdot s_i + \sum e_i$.

While this is a very promising idea, it has a few (correctable!) drawbacks.
1) The setup process to generate $s_{\sum}$ must be repeated each time the party's release data.
2) This setup process can be expensive and expensive to verify (i.e. use verifable variants of MPC in the setup).



\newcommand{\summedSK}{s_{\sum}}

\section{Proposed Solution}
We will specifically focus on improving privacy and integrity in decentralized machine learning where data is additively used as in Ref.~\cite{stevens2021efficientdifferentiallyprivatesecure}.
Instead of using LWE samples as one-time pads though, we will use the output of HPRFs which can in turn be constructed from LWE/ LWR~\footnote{Specifically we require \emph{weak} HPRFs so that the keys are leakage resilient}.
The key advantage is that we only need to do a verifiable setup process \emph{once} for each party and then the party can release data essentially an unlimited number of times (up to a polynomial/ sub-exponential number of times depending on assumptions).
We sketch the protocol in \cref{fig:protSimp}.

\begin{figure}[H]
	\begin{mdframed}
		Setup: \begin{itemize}
			\item Each party $i \in [n]$ generates a secret key $s_i.$
			\item Using MPC, the parties compute $\summedSK = \sum s_i.$
		\end{itemize}
		Aggregation for time step $t$: \begin{itemize}
			\item Each party $i$ calls $\vec{r}_i = F(s_i, t)$  where $F$ is the HPRF.
			\item Each party generates noise $\vec{x}_i$ with standard deviation $O(1/n)$ and publishes data $\overrightarrow{ct}_i = \vec{v}_i + \vec{r}_i + \vec{x}_i$ 
			\item To decode the data, a third party does $\sum_i \overrightarrow{ct}_i - F(\summedSK, t) = \sum_i \vec{v}_i + \sum_i \vec{x}_i$
		\end{itemize}
	\end{mdframed}
	\caption{Outline of the protocol without any verifiability}
	\label{fig:protSimp}
\end{figure}
\sam{Might need to make the standard deviation larger $O(1/\sqrt{n})$ to make it work as per Prop. 3.1. Should it also
depend on the number of honest parties???}

\subsubsection*{Correctness}
\lev{TODO: mostly straightforward, but need to be careful due to wrap around}

\subsubsection*{Soundness and Privacy}

Before proving privacy, we have to recall a theorem about discrete Gaussian noise.


\begin{proposition}[Sum of Discrete Gaussians, \cite{kairouz2021distributed}]
	\label{prop:disc}
	Let $\sigma \geq \frac{1}{2}$.
	Let $X_{i,j} \sim \mathcal{N}_\mathbb{Z}(0, \sigma^2)$ independently for each $i$ and $j$.
	Let $X_i = (X_{i,1}, \dots, X_{i,d}) \in \mathbb{Z}^d$.
	Let $Z_m = \sum_{i=1}^m X_i \in \mathbb{Z}^d$. Then, for all $\Delta \in \mathbb{Z}^d$ and all $a \in [1, \infty)$,
	\[
		D_a(Z_m \parallel Z_m + \Delta) \leq \frac{a \|\Delta\|_2^2}{2m\sigma^2} + \tau_{m,\sigma} d,
	\]
	where $\tau_{m,\sigma} := 10 \cdot \sum_{k=1}^{m-1} e^{-2\pi^2\sigma^2k/(k+1)}$.
\end{proposition}


\begin{theorem}[Preservation of Differential Privacy]
	Let $v^\ast \coloneqq \max_{i \in[n]} \|\vec{v}_i\|_2^2$.
	Suppose there are at least $m = \theta n$ honest parties where $\theta = \frac{a v^\ast}{2\sigma^2 \epsilon}$ for some $\epsilon > 0$ and $a > 1$.
	Assuming that the HPRF is secure and the usage of the fixed-point representation trick in Ref.~\cite{stevens2021efficientdifferentiallyprivatesecure}, the protocol outlined in \cref{fig:protSimp} is $(a, \epsilon)$-RDP.
	% \lev{TODO: exact parameters!}
	\sam{The exact parameters provided assumes that $\tau d$ from Prop. 3.1 is vanishingly small. In Ref.~\cite{stevens2021efficientdifferentiallyprivatesecure}, they make the same assumption because of the fixed-point representation of the noisy gradients which they used.}
\end{theorem}
\begin{proof}[Proof Sketch]
	By \cref{prop:disc}, we have that as long as long as $\theta n$ parties are not malicious and add discrete Gaussian noise honestly, the sum of the noisey shares satisfies Renyi differential privacy.
	Then, by the data processing inequality, any function of the sum of the honest shares also satisfies Renyi differential privacy.
\end{proof}

\begin{theorem}[Liveness]
	The protocol offers no liveness guarantees: any $1$ party can prevent the protocol from completing.
\end{theorem}
\begin{proof}[Proof Sketch]
	To see why this is the case, any party, $i$, can simply abort and not publish anything.
	Then, without $s_i$, $A \cdot s_j + e_j$ for $j \neq i$ remains indistinguishable from random.
\end{proof}


\section{Adding in Verifiability}
Because each party's evaluation of the HPRF is relatively inexpensive and simple (via a matrix multiply and rounding), we can use zk-SNARKs to verify that the party's published data is correctly masked relative to the HPRF, that the underlying data is within a certain range (too ensure that the data is not poisoned), and that the noise is not too large (also preventing poisoning).
As a nice benefit, adding verifiability gives a sort of ``traitor-tracing'' mechanism to the system for $N - 1$ parties while assuming that we have $N /2$ honest parties.
\footnote{Traitor tracing is specifically useful when using ``carrot and stick'' incentives to ensure that parties are honest (such as proof-of-stake).}

%Though not within scope, given the uniformity of each proof, a proof aggregator could also be used to enhance practicality.
We sketch the protocol in \cref{fig:prot}.


\begin{figure}[H]
	\begin{mdframed}
		Setup: \begin{itemize}
			\item Each party $i$ generates a secret key $s_i$
			\item Using MPC, the parties compute $\sum s_i = \summedSK$
			\item Each party releases $Comm(s_i)$ where $Comm$ is a randomized commitment function
		\end{itemize}
		Aggregation for time step $t$: \begin{itemize}
			\item Each party $i$ calls $b_i = F(s_i, t)$  where $F$ is the HPRF
			\item Each party generates (quantized) Gaussian noise $\eta_i \in \Z^n$ with standard deviation $O(1/n)$.
			\item  For data $v_i$ and $PRF$ output $b_i$,
				each party publishes data $ct_i = \lfloor Q \cdot (v_i + b_i) \rfloor  + \eta_i$ for quantization/ offset parameter $Q$.
			\item Each party publishes a proof that (1) $b_i$ is generated with key $s_i$ relative to $Comm(s_i)$, (2) $b_i = F(s_i, t)$, and (3) $v_i + \eta_i$ are within some bounds specified by the protocol
			\item To decode the data, a third party checks all the proofs and does $\left(\sum_i ct_i - F(s_{\sum}, t)\right) = \left(\sum_i v_i + \sum \eta_i\right) / Q$
		\end{itemize}
	\end{mdframed}
	\caption{Outline of the final protocol}
	\label{fig:prot}
\end{figure}

% Delta = sum of v_is
% Everything has to be a vector

% Define theta to be some fraction that should be related to epsilon and noise parameter
% CHange n in Prop 3.1 to honest parties



% \section{Background and Proposed Solution}
% We make extensive use of the following cryptographic primitives: \emph{Zero-knowledge proofs} (ZKPs)and \emph{homomorphic pseudo-random functions} (HPRFs).
% We also use standard primitives such as commitments.



%Introduced by Goldwasser, Micali, and Rackoff in the 1980s, ZKPs rely on complex mathematical principles to achieve this privacy-preserving verification. 
%ZKPs are particularly useful in applications where privacy is paramount, such as identity verification, blockchain transactions, and secure voting protocols.

%In cryptographic protocols, ZKPs are significant because they allow secure authentication and verification without compromising the confidentiality of sensitive information. 
%For example, in blockchain systems, ZKPs can verify transactions without revealing transaction details, ensuring both security and privacy in decentralized environments. 
%Additionally, ZKPs can be used in privacy-preserving computation, where multiple parties can collaboratively compute a function over their inputs without revealing those inputs to each other.
%This makes ZKPs a powerful tool in enhancing privacy and security across various cryptographic applications, from secure messaging to confidential financial transactions.

%Noise generation is crucial for privacy-preserving mechanisms like \emph{differential privacy} (DP) because it masks the contribution of individual data points in statistical analysis, ensuring that sensitive information is protected~\cite{dwork2006differential}. 
%By adding random noise from distributions such as Laplace or Gaussian, DP limits the ability of an adversary to infer specific data, even when multiple queries are made. 
%This balance between privacy and accuracy is often carefully calibrated with more noise enhancing privacy but reducing the utility of the results . 
%In certain advanced scenarios, correlated noise, where noise is not independent but follows specific patterns, can improve privacy and utility, though verifying its correct implementation becomes more challenging .
%
%In ZKPs, noise generation plays a complementary role by ensuring that sensitive data remains hidden while still allowing verification of computations.
%ZKPs allow a prover to demonstrate that noise has been correctly added according to privacy standards, without revealing the noise values or underlying data, preserving confidentiality. 
%This is particularly important in complex systems like federated learning (FL) or secure multi-party computation (MPC), where verification of proper noise addition across multiple parties is required for privacy guarantees. 
%In such systems, noise acts as both a tool for obfuscation and a means to provide proof validation, ensuring that privacy is upheld throughout the computation.
%
%\emph{Pseudo-correlated generators} (PCGs) are cryptographic tools designed to generate large amounts of correlated randomness from short, shared seeds between parties. 
%They are particularly useful in scenarios like secure multi-party computation (MPC), where secure, random correlations (such as oblivious transfer or vector oblivious linear evaluation) can reduce communication costs and improve computational efficiency. 
%Instead of having to continuously exchange randomness or communicate for every correlation instance, parties can precompute short, shared seeds and then expand these seeds locally into long pseudorandom strings, which emulate the desired correlated randomness. 
%This allows for efficient preprocessing phases in cryptographic protocols, which is crucial for scalability.
%
%In the context of data privacy, PCGs provide an efficient and secure way to introduce randomness needed for privacy-preserving operations without extensive communication overhead. 
%By allowing local expansion of correlated randomness, PCGs ensure that privacy guarantees, such as differential privacy, can be achieved efficiently even in complex protocols like MPC. 
%For example, in privacy-preserving machine learning or data analysis, PCGs can be used to generate the required randomness for adding noise to datasets, ensuring that individual data points are protected while enabling useful statistical analysis.

%\section{Background and Related Work}
%
%Noise is a fundamental component in many cryptographic primitives, serving to enhance security and privacy by obfuscating sensitive data. 
%In \emph{homomorphic encryption}~\cite{gentry2009fully}, noise is introduced during encryption to allow computations on ciphertexts without revealing the underlying plaintext. 
%While noise grows with each operation, it must be carefully managed to prevent decryption failure. 
%In DP, noise is added to statistical outputs to obscure individual contributions to a dataset, ensuring privacy even in the presence of multiple queries~\cite{dwork2006differential}. 
%Similarly, in \emph{Learning with Errors} (LWE)~\cite{regev2009lattices}, noise is central to the hardness assumption, where the challenge is to solve linear equations obscured by random noise, which provides post-quantum security.
%Each of these canonical works highlights how noise is not only a tool for privacy but also for achieving computational security in cryptographic systems.

%\subsubsection*{Differential Privacy}
%In this section, we draw heavily from a recent survey on Local Differential Privacy (LDP)~\cite{yang2023local}.
%
%There are a few different ways to formally describe differntial privacy.
%For simplicity, we will use the $(\eps, \delta)$-differential privacy definition, though our ideas can be extended to other definitions.
%
%Intuitively, we can think of differential privacy as being a sort of ``2-wise'' independence condition on the input data: if we replace \emph{one} of the user's piece of data with some other possible piece of data, then differential privacy gaurentees that an adversary \emph{cannot} distinguish between the two datasets up to some probabality.
%
%Though not cryptographic in nature, differential privacy offers a notion of ``plausible-deniability'' which can also be generalized to a more $k$-wise independent setting \lev{TODO: cite?}.
%
%We have two flavors of differential privacy: local and non-local (or centralized) which is classical differential privacy.
%
%
%\lev{IDK if this is the actual defn}
%\begin{definition}[$(\eps, \delta)$ Differential Privacy,~\cite{Bassily_2015}]
%	Let $\eps > 0$ and $\calX$ be the domain of the data.
%	Then, a randomized algorithm, $\calM : \calX^n \rightarrow \calM(\calX)$ which is applied to a dataset of $n$ elements satisfies $(\eps, \delta)$-differential privacy if and only if for all $\vec{x} \in \calX^n$ and $\vec{x}'$ where $\vec{x'}_i \neq \vec{x}_i$ for a specific $i$,
%	\[
%		\Pr[\calM(\vec{x}) \in S] \leq e^\eps \Pr[\calM(\vec{x}') \in S] + \delta
%	\]
%	for any possible $S \subseteq \text{Range}(\calM)$.
%\end{definition}
%
%Note that in the above, the randomized algorithm $\calM$ is applied globally.
%We can also deffine differential privacy locally:
%
%\begin{definition}[$(\eps, \delta)$-Local Differential Privacy,~\cite{Bassily_2015}]
%	Let $\eps > 0$ and $\calX$ be the domain of the data.
%	Then, a randomized algorithm, $\calM_{loc} : \calX \rightarrow \calM_{loc}(\calX)$ which is applied to the data independently satisfies $(\eps, \delta)$-local differential privacy if and only if for all pairs $x, x' \in \calX$
%	for any possible $S \subseteq \text{Range}(\calM_{loc})$, we have
%	\[
%		\Pr[\calM_{loc}(x) \in S] \leq e^\eps \Pr[\calM_{loc}(x') \in S] + \delta.
%	\]
%\end{definition}
%
%Non-local and local differential privacy are usually achieved by adding noise to a dataset.
%Though many mechanisms exist, we will outline the \emph{Laplace Mechanism} for non-local differential privacy.
%
%Let $f: \mathcal{X}^n \rightarrow \R^d$ be some query associated with the all of the user's data (e.g. learning the mean of the data).
%Then, as outlined in \cite{Bassily_2015}, for any $x$,
%\[
%	\calM(\vec{x}) = f(\vec{x}) + N
%\]
%where $N \sim Lap(0, \Delta f/ \eps)^{d}$ where $\Delta f$ is the $\ell_1$ sensitivity of $f$.
%Analogously, we can think of local-differential privacy as \[
%	\calM(x) = f(x) + N.
%\]
%
%\textbf{Problem with Differential Privacy:} Non-local differential privacy requires trusting some centralized aggregator.
%For example, in federated learning, each party would have to send there updated gradients \emph{in the clear} to an aggregator.
%Recent work \lev{TODO CITE} shows that an aggregator can almost perfectly recover the users' data from the gradient delta which could yeild a large privacy violation.
%On the other hand, local differential privacy preserves users' privacy but the quality of the data degrades with an increase in the number of users.
%Specifically, imagine trying to compute the sum of a dataset with $n$ users.
%If each user adds Laplacian noise with a constant standard deviation, $\Delta$, then the sum of the data (calculated by adding together all of the noisy inputs from each user) has Laplacian noise with paramater $n \Delta$!
%
%\subsubsection*{Cryptographic Approaches}
%The other primary privacy preserving mechanism in machine learning is multi-party computation (MPC \lev{CITE}), which is known as as secure multi-party computation (SMPC) in machine learning to disambiguate from massive parallel computation.
%We draw from a recent survey on SMPC~\cite{zhou2024secure}.
%Given that our work is focused on differential privacy, we will not formally define SMPC here.
%Rather, SMPC is a cryptographic protocol which allows multiple parties to compute a function on their private data without revealing their data to each other.
%SMPC can also be integrated with other cryptogrpahic techniques (such as cut-and-choose type proofs, \lev{CITE}) to provide additional security guarantees: such as integrity checks on the data.
%
%Unfortunately, SMPC is not a panacea.
%Specifically, SMPC, especially with \emph{integrity checks on the input data}, is very slow and cannot scale to large datasets and hard computations.
%Moreover, practical implementations require a lot of interaction between the parties, which can be a bottleneck in some applications \cite{zhao2019secure}.
%
%\subsubsection*{Combining MPC, LWE, and Differential Privacy}
%Ref.~\cite{stevens2021efficientdifferentiallyprivatesecure} introduced the idea of combining Learning with Error (LWE) with differential privacy for aggregation.
%\footnote{Not too brag, but we discovered this idea independently prior to finding the referenced paper}
%The key idea is to use an LWE sample for each party $i$, $b_i = A \cdot s_i + e_i$ where $s_i$ is secret and $e_i$ is private noise, to mask each users' data.
%Then, using MPC or a similar protocol, the users can publish $s_{\sum} = \sum s_i$ which can then be used to compute an approximation of the aggregated one-time pads via $A \cdot s_{\sum} \approx \sum b_i = \sum A \cdot s_i + \sum e_i$.
%
%While this is a very promising idea, it has a few (correctable!) drawbacks.
%1) The setup process to generate $s_{\sum}$ must be repeated each time the party's release data.
%2) This setup process can be expensive and expensive to verify (i.e. use verifable variants of MPC in the setup).
%

%\section{Proposed Solution}
%We will specifically focus on improving privacy and integrity in decentralized machine learning where data is additively used as in Ref.~\cite{stevens2021efficientdifferentiallyprivatesecure}.
%Instead of using LWE samples as one-time pads though, we will use homomorphic pseudo-random functions (HPRFs) \lev{TODO cite} which can in turn be consturcted from LWE.
%\footnote{Specifically we require \emph{weak} HPRFs so that the keys are leakage resilient}
%The key advantage is that we only need to do a verifiable setup process \emph{once} for each party and then the party can release data essentially an unlimited number of times (up to a polynomial/ sub-exponential number of times depending on assumptions).
%
%\textbf{Adding in Verifiability:} Because each pary's evaluation of the HPRF is relatively inexpensive and simple (via a matrix multiply and rounding), we can use ZK-SNARKs to verify that the party's published data is correctly masked relative to the HPRF, that the underlying data is within a certain range (too ensure that the data is not poisoned), and that the noise is not too large (also preventing poisoning).
%Indeed adding verifiability gives a sort of ``traitor-tracing'' mechanism to the system for $N - 1$ parties while assuming that we have $N /2$ honest parties.
%Though not within scope, given the uniformity of each proof, a proof aggregator could also be used to enhance practicality.
%For simplicity, we will use RISC-0 as our ZK-SNARK of choice \lev{TODO cite}.
%We sketch the protocol in \cref{fig:prot}.
%
%\begin{mdframed}
%	Setup (preferably ran within verifiable MPC, though not in scope) \begin{itemize}
%		\item Each party $i$ generates a secret key $s_i$
%		\item Using MPC, the parties compute $\sum s_i = s_{\sum}$
%		\item Each party releases $H(s_i)$ where $H$ is a hash function
%	\end{itemize}
%	Aggregation for time step $t$ \begin{itemize}
%		\item Each party $i$ calls $b_i = HPRF(s_i, t)$ 
%		\item Each party generates noise $\eta_i$ with standard deviation $O(1/n)$ and publishes data $ct_i = v_i + b_i + \eta_i$ as well as a proof that $b_i$ is generated with key $s_i$ relative to $H(s_i)$ and that $b_i = HPRF(s_i, t)$ and that $v_i + \eta_i$ are within some bounds
%		\item To decode the data, a third party checks all the proofs and does $\sum_i ct_i - HPRF(s_{\sum}, t) = \sum_i v_i + \sum \eta_i$
%	\end{itemize}
%	% \caption{Outline of the Final Protocol}
%	\label{fig:prot}
%\end{mdframed}

\bibliographystyle{alpha}
\bibliography{bib/ref}

\end{document}



